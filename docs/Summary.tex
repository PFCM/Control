\documentclass[11pt]{article}
\usepackage[a4paper,margin=2.7cm]{geometry}


\usepackage{tgcursor}
%\usepackage{tgheros}
\usepackage[T1]{fontenc}

%\renewcommand{\familydefault}{\sfdefault}

\usepackage{enumerate}
\usepackage[sc,compact]{titlesec}
\usepackage{bigfoot}

\begin{document}
\raggedright
\section*{Summary}
\begin{enumerate}[\bf1.]
	\item Server starts running
	\begin{enumerate} [\bf a.]                          
		\item Searches \texttt{Instruments} directory, attempts to load files (ignores directories)
		\item Constructs list of instruments from files
		\begin{enumerate}
			\item MidiInstrument class sets up MIDI output and translation from OSC
			\item Base class Instrument sets up OSC listeners according to what is defined in file.
			\item All instruments have two default messages they expect if nothing is specified -- \texttt{/\textit{<name>}/note,ii} and 
				 \texttt{/\textit{<name>}/control,ii} 
		\end{enumerate}
		\item Starts listening for new clients
	 \end{enumerate}
	 \item Client starts 
	 \begin{enumerate} [\bf a.]
	 	\item Sends \texttt{/system/addme,si} to Server (with own hostname and port)
		\item Server responds with a series of \texttt{/system/instruments/add,s} messages which list the instruments constructed earlier by name.
		\item Any instruments with possible messages beyond the basic two send
			 \texttt{/system/instruments/extend,ssi} to the client
			which contain the name of the instrument, the pattern for the message and the MIDI status byte to transform.
		\item All instruments send any information they know about themselves in \texttt{/system/instruments/note,ss} messages where the first 
			string is the name of the instrument and the second is a note about the instrument, probably defined in the data file. \footnote{Confusion
			between \texttt{/\textit{<name>}/note,ii} and \texttt{/system/instrument/note,ss} should be avoidable given the different typetags and the 
			\texttt{/system} prefix, although it is an unfortunate homonym.} The notes get displayed by the client to give the user any information
			the instrument's designer feels useful.
		\item Client uses this data to construct a table of MIDI input to OSC output and prints details about the instruments connected to the server
			to the console.
		\item Client listens for MIDI input on specified port and translates appropriately.
	 \end{enumerate}
\end{enumerate}
\end{document}