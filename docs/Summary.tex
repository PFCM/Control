\documentclass[11pt]{article}
\usepackage[a4paper,margin=2.7cm]{geometry}
\usepackage{enumerate}
\usepackage[sc,compact]{titlesec}
\usepackage{tgcursor}
\usepackage[T1]{fontenc}
\usepackage{bigfoot}

\begin{document}
\raggedright
\section*{Summary}
\begin{enumerate}[\bf1.]
	\item Server starts running
	\begin{enumerate} [\bf a.]
		\item Searches \texttt{Instruments} directory, attempts to load files (ignores directories)
		\item Constructs list of instruments from files
		\begin{enumerate}
			\item MidiInstrument class sets up MIDI output and translation from OSC
			\item Base class Instrument sets up OSC listeners according to what is defined in file.
			\item All instruments have two default messages they expect if nothing is specified -- \texttt{/\textit{<name>}/note,ii} and 
				 \texttt{/\textit{<name>}/control,ii} 
		\end{enumerate}
		\item Starts listening over OSC for new clients
	 \end{enumerate}
	 \item Client starts up
	 \begin{enumerate} [\bf a.]
	 	\item Sends \texttt{/system/addme,si} to Server (with hostname and port)
		\item Server responds with a series of \texttt{/system/instruments/add,s} messages which list the instruments constructed earlier by name.
		\item Any instruments with possible messages beyond the basic two send
			 \texttt{/system/instruments/extend,ssi} to the client
			which contain the name of the instrument, the pattern for the message and the MIDI status byte to transform.
		\item Client uses this data to construct a table of MIDI input to OSC output
		\item Client listens for MIDI input on specified port and translates appropriately.
	 \end{enumerate}
\end{enumerate}
\end{document}