\documentclass[12pt]{article}
\usepackage[a4paper,margin=1cm]{geometry}
\usepackage{syntax}
\usepackage{enumerate}
\usepackage{float}
\floatstyle{boxed}
\restylefloat{figure}

\begin{document}

This section defines the data files which specify what instruments are available. The precise definition of the file syntax is provided in Extended 
Backaus-Naur form in figure \ref{ebnfspec}. 
It must 
\begin{enumerate} [a)]
\item{begin with a ``type'' field consisting of \texttt{type=} followed by a description of a type which the system knows. This can be
a constant, shorthand for one of the default generic types (at the moment: \texttt{MIDI}) or a file name (ending in \texttt{.ck}). In the case of a file name
the system will look for that file in the default directory (\texttt{src/}) and attempt to add it to the virtual machine and create an object with the same
name as the file (minus the \texttt{.ck} suffix).}
\item{follow this with 0 or more translation lines.}

\end{enumerate}

\begin{figure} [htp]
\setlength{\grammarindent}{12em}
\begin{grammar}
	<file> ::= <type> <linebreak> [ <translation> <linebreak>] + 
	
	<type> ::= ``type=''<type string> <linebreak>
	
	<type string> ::= ``MIDI'' \alt <filename>
	
	<filename> ::= <name>.ck
	
	<name> ::= `[a-zA-Z][a-zA-Z0-9_]*'

\end{grammar}


\caption[Grammar]{Data file grammar}
\label{ebnfspec}
\end{figure}

\end{document}
