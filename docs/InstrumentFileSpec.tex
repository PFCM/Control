\documentclass[12pt]{article}
\usepackage[a4paper,margin=2.7cm]{geometry}
\usepackage{syntax}
\usepackage{enumerate}
\usepackage{float}
\floatstyle{boxed}
\restylefloat{figure}
\usepackage{tgcursor}
\usepackage[T1]{fontenc}

\begin{document}

This section defines the data files which specify what instruments are available. The precise definition of the file syntax is provided in 
Backaus-Naur form below. 
It must 
\begin{enumerate} [\bfseries a)]
\item{begin with a ``type'' field consisting of \texttt{type=} followed by a description of a type which the system knows. This can be
a constant, shorthand for one of the default generic types (at the moment: \texttt{MIDI}) or a file name (ending in \texttt{.ck}). In the case of a file name
the system will check the first line for \texttt{// type=}. It will then attempt to match the remainder of the line to a constant defined in the \texttt{Server.ck}
file, at which point it should add the file to the virtual machine and instantiate an object. Note that this code will have to be added to \texttt{Server.ck}
when a new instrument is added by a \texttt{.ck}.}
\item{follow this with a name, specified by \texttt{name=}\textit{name}}
\item{a MIDI file will now look for a port, specified as \texttt{port=}\textit{number or name}} (if the name of the port is specified, it can be in quotes,
 but it is not required)
\item{follow this with 0 or more translation lines. The translation lines are of the form \texttt{input=message=output} where input is the input midi
	 status byte (disregarding channel), message is the OSC message that it becomes and output is the MIDI message sent to the instrument.
	 The osc message will always have the typetag \texttt{ii} with the remaining two input midi bytes (or 0 if it is a smaller message) sent as 
	 arguments. In specifying the output message the special terms \texttt{\$1} and \texttt{\$2} can be used to refer to the first and second arguments
	 of the osc message. The input status byte (and following = ) is optional, omitting it will cause the server to still listen for the specified OSC
	 message and perform the appropriate transform, but will not set up the client to generate the OSC.}

\end{enumerate}

Note that messages that do not begin with the name of the instrument will have it prefixed automatically. Also if the default messages are not specified, 
they will be added.
A file with 0 translation lines would cause the server to instantiate the object specified, and may therefore be of some use.

\subsection{Grammar}
\setlength{\grammarindent}{12em}
\begin{grammar}
	<file> ::= <type> <linebreak> <name> <linebreak> ( <file-element> <linebreak> ) * 
	
	<type> ::= ``type=''<type-string> <linebreak>
	
	<type-string> ::= ``MIDI''
	
	<name> ::= ``name=''<name-string> <linebreak>
	
	<name-string> ::= `[a-zA-Z][a-zA-Z0-9_]*'
	
	<file-element> ::= <midi-port> \alt <translation>
	
	<midi-port> ::= `port=' <midi-port-desc>
	
	<midi-port-desc> ::= `[a-zA-Z][a-zA-Z0-9]*' \alt `[0-9]+'
	
	<translation> ::= [<input-status-byte> `='] <osc-message-desc> `=' <output-message-desc> <linebreak>
	
	<input-status-byte> ::= <midi-stat-byte>
	
	<osc-message-desc> ::= `"'<osc-addr-pat>  <osc-typetag> `"'
	
	<osc-addr-pat> ::= (`/' <osc-string> )+
	
	<osc-typetag> ::= `, ' <osc-type>+
	
	<osc-type> ::= `i' \alt `f' \alt `s' \alt `b'
	
	<osc-string> ::= `[^\\0]+'
	
	<output-message-desc> ::= <midi-message> \alt future message types
	
	<midi-message> ::= <midi-stat-byte> `,' <midi-data-byte> `,' <midi-data-byte>
	
	<midi-stat-byte> ::= 128--255
	
	<midi-data-byte> ::= 0--127 \alt <osc-arg>
	
	<osc-arg> ::= `$'[1-2]

\end{grammar}
\end{document}
