\documentclass[11pt]{article}
\usepackage[a4paper,margin=2.7cm]{geometry}
\usepackage{enumerate}
\usepackage[sc,compact]{titlesec}
\usepackage{tgcursor}
\usepackage[T1]{fontenc}
\usepackage{bigfoot}
\usepackage{tabularx}
\usepackage{hyperref}
\usepackage{microtype}

\begin{document}
\title{Tangle User Manual}
\maketitle

\tableofcontents
\newpage
\section{Quick start instructions}
\begin{enumerate} [\bf 1.]
\item {\large Server}
	\begin{enumerate} [\bf i.]

		\item{Check all the desired instruments are connected correctly.}
		\item{Start up the server. Once it has booted, log in either remotely---
		\begin{verbatim}
			    %> ssh sonics@192.168.33.1
		\end{verbatim}                                 
		or using the login screen. If you log in locally to the server, open a terminal window.}
		\item{Check \texttt{Instruments} directory---
		\begin{verbatim}
			cd ~/Network/Tangle
			ls Instruments/
		\end{verbatim}
		make sure files are present for each of the instruments you intend to use.}
		\item Run \texttt{Master.ck} -- in the shell this is (assuming we are still in the root project directory)
		\begin{verbatim}
			    %> chuck src/Master
		\end{verbatim}
		\item  It will produce a lot of text, double check to make sure no errors were emitted. You could run the 
			  server from the miniAudicle, but this is often less stable over long periods of time.
	\end{enumerate}
\item {\large On your own machine}
	\begin{enumerate} [\bf i.]
		\item {If you want to use your own OSC}
		\begin{enumerate} [\bf a.]
			\item{Double check the address patterns the server printed at the end of its initialisation, you should be good to go.}
		\end{enumerate}
		\item {If you want to use MIDI run \texttt{Client.ck} -- it needs some information to function correctly.
			  The minimum is: }
		\begin{enumerate} [\bf a.]
			\item{An address for the server; hopefully 192.168.33.1 or \texttt{leoadmins-Mac-mini.local} will work
			\footnote{ To find the IP address the server is using for the ethernet either look in the settings or use \verb+ifconfig+
			to find the IPv4 address.}}
			\item{An IP address for the client machine}
			\item Using this information, run \texttt{Client.ck}. The information above must be passed as arguments. For a server at the address 
				192.168.33.1, a client IP address of 192.168.33.3 and MIDI on the port ``IAC Driver 1 Bus 1'' the command would be as follows:
				\begin{verbatim}
					    %> chuck Client.ck:server=192.168.33.1:self=192.168.33.3:\
					            midi=IAC Driver 1 Bus 1
				\end{verbatim}
				A full list of options an be found in section \ref{sec:clientopts}.
		\end{enumerate}
	\end{enumerate}

\end{enumerate}

\section{Start Up Procedure}
\label{sec:startupproc}
What actually happens when the above instructions are followed:

\begin{flushleft}
\begin{enumerate}[\bf1.]
	\item Server starts running
	\begin{enumerate} [\bf a.]                          
		\item Searches \texttt{Instruments} directory, attempts to load files (ignores directories)
		\item Constructs list of instruments from files
		\begin{enumerate}
			\item MidiInstrument class sets up MIDI output and translation from OSC
			\item Base class Instrument sets up OSC listeners according to what is defined in file.
			\item All instruments have two default messages they expect if nothing is specified -- \texttt{/\textit{<name>}/note,ii} and 
				 \texttt{/\textit{<name>}/control,ii} 
		\end{enumerate}
		\item Starts listening for new clients
	 \end{enumerate}
	 \item Client starts 
	 \begin{enumerate} [\bf a.]
	 	\item Sends \texttt{/system/addme,si} to Server (with own hostname and port)
		\item Server responds with a series of \texttt{/system/instruments/add,s} messages which list the instruments constructed earlier by name.
		\item Any instruments with possible messages beyond the basic two send 
			 \texttt{/system/instruments/extend,ssi} to the client
			which contain the name of the instrument, the pattern for the message and the MIDI status byte to transform.
		\item All instruments send any information they know about themselves in \texttt{/system/instruments/note,ss} messages where the first 
			string is the name of the instrument and the second is a note about the instrument, probably defined in the data file.\footnote{Confusion
			between \texttt{/\textit{<name>}/note,ii} and \texttt{/system/instrument/note,ss} should be avoidable given the different typetags and the 
			\texttt{/system} prefix, although it is an unfortunate homonym.} The notes get displayed by the client to give the user any information
			the instrument's designer feels useful.
		\item Client uses this data to construct a table of MIDI input to OSC output and prints details about the instruments connected to the server
			to the console.
		\item Client checks if any latency calibration has been specified, if so sends to server. Blocks until server indicates it is complete.
		\item Client checks if any test patterns have been asked for, if so sends to server, blocking until notified of completion.
		\item Client listens for MIDI input on specified port and translates appropriately.
	 \end{enumerate}
\end{enumerate}
\end{flushleft}

\section{Client.ck options}
\label{sec:clientopts}
Options for \texttt{Client.ck} are specified via a colon separated list of arguments. All arguments are specifed in the form \texttt{<key>=<value>}. A full list of options
is in table \ref{tab:clientopts}.

\begin{table} [h]
\caption {\texttt{Client.ck} options}
\label{tab:clientopts}
\begin{tabularx} {\textwidth} { | >{\bfseries}r | X | X |}
\hline
key			& values											& Description \\
\hline \hline
server		& url or numerical IP in the format AAA.BBB.CCC.DDD & Tells the client the IP address of the server. \\
\hline
self			& same as for server							    & Gives the server a return address to send information about connected instruments. \\
\hline
in 			& an integer								    & Port on which Client listens for communication from server (default 50001). \\
\hline
out		         & an integer								    & Port which server is listening on (default 50000). \\
\hline
midi			& an integer or the name of a MIDI port			    & MIDI port on which the client listens, defaults to 0. It is better to use a name as the order
														of these can shift. Available MIDI ports can be found by running \texttt{chuck --probe}. \\
\hline
test			& a comma separate list of names of instruments (ignores case) & Tells the client whether or not to ask the server to run a test pattern, and if so on
														which instruments. An empty string or \texttt{none} will not cause any tests, \texttt{all} will
														ask the server to test all instruments connected. \\
\hline
delay		& a comma separate list of names, as for test. 		    & Tells the client whether or not to ask the server to begin the delay calibration process. 
														Special values \texttt{on} or \texttt{off} tell the server to calibrate all or no instruments
														respectively. \\
\hline
\end{tabularx}

\end{table}

An example of a likely command to run the client might then be:
\begin{verbatim}
	%> chuck Client.ck:self=192.168.33.3:server=192.168.33.1:\
			midi=IAC Driver 1 Bus 1:test=all:delay=on
\end{verbatim}

\section{Adding New Instruments}
The server discovers instruments by searching the \texttt{Tangle/Instruments}. Each instrument requires a file which tels the server how to talk to it. This can be a 
configuration file to be read in or a ChucK source file which contains a sub-class of Instrument.

\subsection{MIDI Instrument Config Files}


\subsection{Subclassing Instrument}

\end{document}