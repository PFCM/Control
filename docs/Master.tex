\documentclass[11pt]{article}
\usepackage[a4paper,margin=2.7cm]{geometry}
\usepackage{enumerate}
\usepackage[sc,compact]{titlesec}
\usepackage{tgcursor}
\usepackage[T1]{fontenc}
\usepackage{bigfoot}

\begin{document}
\section{Master.ck}
\texttt{Master.ck} is the file which starts the server running. In order to do this it adds all the required files to the virtual machine. In order for the server to work 
properly with any possible number of new types of instruments it also adds any \texttt{.ck} files in the \texttt{instruments} directory. To make it easy to move
 instruments to and from the \texttt{Unconnected} directory, \texttt{Master.ck} searches recursively through all subdirectories of \texttt{Instruments} adding 
everything it finds. The order in which the files are added to the virtual machine is below.

\begin{enumerate} [\bf 1.]
	\item\texttt{Util.ck}
	\item\texttt{Parser.ck}
	\item\texttt{MIDIDataByte.ck}
	\item\texttt{MIDIMessageContainer.ck}
	\item\texttt{Instrument.ck}
	\item\texttt{MIDIIInstrument.ck}
	\item\texttt{MultiStringInstrument.ck}
	\item Instruments found in \texttt{Instruments} and subdirectories. This is done with a pre-order depth first search.\footnote{All files with a \texttt{.ck} 
	extension are added in the current directory (beginning in \texttt{Instruments}) and the process is repeated for each subdirectory in turn. Note that if you
	want to add a new type of abstract instrument or extend a custom instrument, you may need to pay attention to this order and probably edit 
	\texttt{Master.ck} to ensure everything is added in the correct order.}
\end{enumerate}

\end{document}