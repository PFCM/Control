\documentclass[11pt]{article}
\usepackage[a4paper,margin=2.7cm]{geometry}
\usepackage{enumerate}
\usepackage[sc,compact]{titlesec}
\usepackage{tgcursor}
\usepackage[T1]{fontenc}
\usepackage{bigfoot}

\begin{document}
\section{Steps}

\subsection{ChucK / ssh}
\begin{enumerate} [\bf 1.]
\item {\large Start the server}
	\begin{enumerate} [\bf i.]

		\item{Check all the desired instruments are connected correctly.}
		\item{Start up the server. Once it has booted, log in either remotely---
		\begin{verbatim}
			    %> ssh sonics@192.168.33.1
		\end{verbatim}                                 
		or using the login screen. If you log in locally to the server, open a terminal window.}
		\item{Check \texttt{Instruments} directory---
		\begin{verbatim}
			cd Documents/RobotNetwork 
			ls Instruments/
		\end{verbatim}
		make sure files are present for each of the instruments you intend to use.}
		\item Run \texttt{Master.ck} -- in the shell this is (assuming we are still in the root project directory)
		\begin{verbatim}
			    %> chuck src/Master
		\end{verbatim}
		\item  It will produce a lot of text, double check to make sure no errors were emitted. You could run the 
			  server from the miniAudicle, but this is often less stable over long periods of time.
	\end{enumerate}
\item {\large On your own machine}
	\begin{enumerate} [\bf i.]
		\item {If you want to communicate over OSC}
		\begin{enumerate} [\bf a.]
			\item{Double check the address patterns the server printed at the end of its initialisation, you should be good to go.}
		\end{enumerate}
		\item {If you want to use MIDI run \texttt{Client.ck} -- it needs some information to function correctly.
			  The minimum is: }
		\begin{enumerate} [\bf a.]
			\item{An address for the server; hopefully 192.168.33.1 or \texttt{leoadmins-Mac-mini.local} will work
			\footnote{ To find the IP address the server is using for the ethernet either look in Network Preferences or use \verb+ifconfig en0 | grep inet+
			which should print the IPv4 and IPv6 addresses, the former of which is what we are after.}}
			\item{An address for the client machine, you could use a numerical address or \textit{<your-computer-name>}.local if it is a Mac.}
			\item{You can also specify the MIDI port, it will default to the first in the list. This can be specified by name or number.}
			\item{Also optional are send and receive ports for the transmission -- by default the server listens to port 50000 and will communicate back
				on any port specified by the Client.}
			\item{Using this information, run \texttt{Client.ck}. The information above must be passed as arguments. For a server at the address 
				192.168.33.1, a client IP address of 192.168.33.3 and MIDI on the port ``IAC Driver 1 Bus 1'' the command would be as follows:
				\begin{verbatim}
					    %> chuck Client.ck:192.168.33.1:self=192.168.33.3:\
					            midi=IAC Driver 1 Bus 1
				\end{verbatim}
				For more detailed examples see the documentation for \texttt{Client.ck}.}
		\end{enumerate}
	\end{enumerate}

\end{enumerate}

\end{document}