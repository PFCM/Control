\documentclass[../main.tex]{subfiles}
\begin{document}

This is the main file for the Server. It initialises the attached instruments and triggers them to send information about themselves to new clients.
The server discovers which instruments are available by looking in the \texttt{Instruments} directory, ignoring all subdirectories. When it discovers
a file it attempts to load the appropriate instrument based on the first line of the file. The Server listens for UDP traffic on port 50000 by default, this can be 
changed by altering \texttt{Server.ck}. The server initialises instruments to share a reference to a send \texttt{OscRecv} object, all instruments listen on the same 
port. The handling of instrument input is handled entirely within \texttt{Instrument} and its subclasses. 

When initialising instruments, the server reads the first line of the file to determine what sort of file it is. It then constructs and instance of the appropriate subclass
and calls its \texttt{init()} method, passing the the \texttt{OscRecv} and the file object. The instrument then handles configuring itself based on the file, with the 
return value of \texttt{init()} indicating whether or not it was successful.

Once all the instruments have been loaded, the server shred begins listening for new clients. A new client announces itself by sending a \texttt{/system/addme,si} 
OSC message to the server, where the two arguments are the hostname or IP address and the port for return communication. Using this information, the client 
iterates through the list of instruments and calls each ones \texttt{sendMethods()} and \texttt{sendNotes()} methods. These messages all take an \texttt{OscSend}
object as their sole argument which they use to send relevant data describing the instrument back to the client.



\end{document}